%=====================%
% uitmthesis Settings %
%=====================%

%==========================%
% For measuring length     %
% Disable when not needed! %
%==========================%
% \usepackage[showframe,pass]{geometry}
% \usepackage[grid,
%  gridcolor=red!20,
%  subgridcolor=green!20,
%  gridunit=mm]{eso-pic}
% % Add lipsum dummy text.
\usepackage{lipsum}

%% Mathematics related
\usepackage{amsmath,amsfonts,amssymb,amsthm}

%% Images
\usepackage{graphicx}
% define images locations.
\graphicspath{{mainmatter/images/},{appendices/images/}}
\usepackage{subcaption}
\usepackage{float}
%% References Type
%\usepackage[numbers]{natbib} % enable to use IEEEtran, disable for apalike2
%\usepackage{natbib} % enable to use apalike2, disable for IEEEtran
%% To use IEEEtran, in settings.tex, use \usepackage[numbers]{natbib}
%% To use IEEEtran, need to replace \citet to \cite or \citep everywhere.
\bibliographystyle{apalike2} %IEEEtran,apalike2.
%\bibliographystyle{IEEEtran}
%\setlength{\bibhang}{0.5in} % set the length of the bibliography hanging indent
\usepackage{csquotes}
%% APA7: style=apa; IEEE: style=ieee
\usepackage[style=apa,backend=biber]{biblatex}
\addbibresource{references/myref.bib}

\usepackage{memhfixc} % add memhfixc to use hyperref package
\usepackage[bookmarksnumbered,bookmarksdepth=4,hidelinks,bookmarks=true,bookmarksopen=true]{hyperref}
\usepackage[numbered]{bookmark}

\usepackage{listings}
\usepackage{xcolor}

\lstset{
    language=Python,            % Language of the code
    basicstyle=\ttfamily\footnotesize, % Basic font size
    keywordstyle=\color{blue},  % Keywords in blue
    stringstyle=\color{red},    % Strings in red
    commentstyle=\color{gray},  % Comments in gray
    numbers=left,               % Line numbers on the left
    numberstyle=\tiny\color{gray}, % Line numbers style
    stepnumber=1,               % Number every line
    showstringspaces=false,     % Don't show spaces in strings
    frame=single,               % Frame around the code
    rulecolor=\color{black},    % Frame color
}


%==================================================================%
% Theorem, corollary, lemma, remark, definition, example, solution %
%==================================================================%
\newtheorem{theorem}{Theorem}[chapter]
\newtheorem{corollary}{Corollary}[theorem]
\newtheorem{lemma}[theorem]{Lemma}
%\theoremstyle{remark}
\newtheorem{remark}{Remark}[chapter]
\theoremstyle{definition}\newtheorem{definition}{Definition}[section]
\newtheorem{example}{Example}[chapter]
%% Solution environment
% from https://tex.stackexchange.com/a/19948/2405
\newenvironment{solution}{%
    \let\oldqedsymbol=\qedsymbol%
    \def\@addpunct##1{}%
    \renewcommand{\qedsymbol}{$\blacktriangleleft$}%
    \begin{proof}[\bfseries\upshape Solution]}%
    {\end{proof}%
    \renewcommand{\qedsymbol}{\oldqedsymbol}}
            
%====================================%
% Settings. Please edit accordingly. %
%====================================%
\title{Comparative Analysis of Speech Detection Models with a Focus on the Japanese Language}
\author{Muhammad Aliff Aiman Bin Zolkifeli}
\date{March 2025}
\datesubmit{July 3, 2025} % viva date for postgraduate / submission date for FYP
\supervisor{Prof. Dr. The Great Supervisor}
\university{Universiti Teknologi MARA}

%% Report (for fyp) / Thesis (for master/phd) / Dissertation
%% Important: Case sensitive!!!
%\reportType{Report} % for FYP
\reportType{Dissertation} % for postgraduate coursework / mixed mode
% \reportType{Thesis} % for postgraduate by research

%% BSc,MSc,MA,MACT,MBA,MIBF,MAS,DBA,PhD,MMus,MEd,LL.M, etc,
%% Important: Case sensitive!!!
%\degreeAcronym{BSc}
%\degreeAcronym{DBA}
%\degreeAcronym{MBA}
\degreeAcronym{MSc}
% \degreeAcronym{PhD}

%% Programme
% \programmeCode{CS952}\programme{Doctor of Philosophy}\programmeT{Mathematics}
%\programmeCode{CS248}\programme{Bachelor of Science (Hons.)}\programmeT{Management Mathematics}
%\programmeCode{CS953}\programme{Doctor of Philosophy}\programmeT{Statistics}
\programmeCode{CDCS707}\programme{Master of Computer Science}\programmeT{Computer Science}
% \programmeCode{CDCS707}\programme{Master of Computer Science}\programmeT{Mathematics}
%\programmeCode{AA901}\programme{Doctor of Business Administration}\programmeT{Educational Management and Leadership}

%% GENDER. male, female i<- important: lowercase!
%\gender{female}
\gender{male}

%% MISC
\studentid{2024152027}
\faculty{Faculty of Computer and Mathematical Sciences}

%============================%
% PDF metadata with hyperref %
%============================%
\makeatletter
\hypersetup{
    pdfinfo={
        Title={\@title},
        Author={\@author},
    }
}
\makeatother
