\chapter{INTRODUCTION}
\label{ch:intro}

\section{Research Background}
\label{sec:intro-bg}

The way people interact with computers has changed rapidly throughout history. Initially, computer instructions were provided through punched cards. Nowadays, giving instruction to computers is simply by using voices. This advancement is made possible by using speech-to-text technology that converting the spoken language into text format \parencite{Xu}. Speech-to-text technologies has already existed in industries such as customer service and medicine. But with the advancement of the machine learning and artificial intelligence, it has made the speech-to-text technology to become more precise and faster \parencite{latif2020}. These advancements have enabled its application across many areas, including transcription services and the development of inclusive tools for individuals with disabilities \parencite{Koenecke2020}.

Although the speech-to-text technology has advance rapidly, it still has challenge like accurately transcribing Japanese language. According to \textcite{Kanno} in "An Introduction to Japanese Linguistics", Japanese language has may words that sound the same but have different meaning and to know which word is being used is based on the current context of the sentence. This is because Japanese language is using syllable-based word formation rather than individual phonemes, it means that the words are created using syllables like "ka", "ki' or "ku" instead of using a single consonant or vowel. Japanese language also using combination of three script with each has its own set of rule makes it harder to convert from spoken language to text. 

With the advancement of machine learning and artificial intelligence, it has significantly improved speech recognition algorithms, enabling them to adapt to language nuances \parencite{xu2023recent}. This study investigates prominent models that is Whisper from Open AI, wav2vec2 by Facebook, and ALMA-7B fine-tuned model from Google Gemma. These models have their pros and cons when transcribing spoken Japanese into text and typically use learning frameworks and undergo training on extensive datasets to improve accuracy in recognizing speech patterns \parencite{ando2021}. It is important to compare Japanese speech recognition systems because most research currently focuses on English or European languages, with limited exploration of how well these systems work with Japanese, especially in casual conversations and real-world contexts.

Only a few studies are comparing the Japanese speech recognition systems which created a gap in this area. It is important to examine how well these models can handle language feature like dialects and how well they able to transcribe spoken language based on accuracy and the speed to convert speech to text. The findings will contribute to the development of Japanese speech recognition technology.


\section{Problem Statement}

Current speech-to-text model are trained on standardized language which might not capture the complexities of Japanese language dialect and informal expression \parencite{imaizumi2022}. This has led to the models cannot perform well when transcribing the conversational Japanese especially when informal words or dialects is being used. Despite the advancements of AI, which significantly increase the quality of text-to-speech model \parencite{Karita2021}, there is still lack of comprehensive evaluation between these models performance with Japanese language. 


The lack of effective speech-to-text solution that tailored for Japanese language has its implication in industries. Industries that relying on speech-to-text technology such as telecommunication, education, technology, may face a problem because ineffective speech recognition can resulting in problems such as misunderstandings and will diminished the user satisfaction \parencite{Sztahó2023}. Additionally, the speech detection technology will not be adopted in industries if it fails to accurately capture the full spectrum of the language, limiting usability and accessibility. Because of that, a study focused on Japanese speech recognition quality is important not only to improve practical outcome but also supports the ongoing advancement in AI field. 


\section{Research Objectives}
\begin{enumerate}
    \item To identify the key requirements for constructing speech-to-text
    model within the context of Japanese language.
    
    \item To analyze speech-to-text models to determine the most effective techniques for reducing Word Error Rate (WER) and transcription latency in Japanese language processing.

    \item To evaluate the WER(Word Error Rate) and the transcription latency of different speech-to-text model when transcribing Japanese formal and informal language.
\end{enumerate}

\section{Research Questions}
\begin{enumerate}
    \item What is the key requirements for constructing speech-to-text model within the context of Japanese language.
    
    \item What is the most effective techniques for reducing Word Error Rate (WER) and transcription latency in Japanese language processing.

    \item How to calculate the performance and effectiveness of different speech-to-text model in context of Japanese language?
\end{enumerate}


\section{Scope of Study}
This study will be focusing on examining the effectiveness of Automatic Speech Recognition (ASR) model for Japanese language speech-to-text technology. The key model is Whisper from OpenAI, wav2vec2 from Facebook's fined-tuned XLSR large language model, and the third model is Chirp, the next generation of Google's speech-to-text models. This study also will analyze the specific linguistic challenges that is unique in the Japanese language. 

Some of the challenges is Japanese language using Syllable-based word formation that created many words that have same sounds but different meaning, which is crucial for the LLM to distinguish which word is being used based on the context of the sentence. On top of that, Japanese language also using three distinct writing system that can be used in one sentence. So, it is also important to see how accurate the LLM transcribe the correct character from the speech. 

After that the evaluation of each model performance based on how well it transcribing both formal and informal Japanese language will be carried out. Formal language is the language that is usually being used in professional setting where informal language is used in the daily life conversation. Then, this study will evaluate the model's performance based on its performance transcribing the different Japanese dialect. 

Aside from the model accuracy, in this study also will be comparing the speed of the model to complete the transcription task. Speed is also one of the important aspects when the model is being used in real-time application where any delay will cause impact on the user experience. By evaluating both the accuracy and processing speed of the model, this study aims to identify which model is high in performance with minimal latency.  


\section{Significance of Study}
This study is aimed to address the gap of effective speech-to-text solution that focusing on Japanese language. Most of the developed LLM is focusing on English language or a generic transcribe model that is developed for multi-language. In this study, the industry implication that caused by the inefficiency in speech to text technology in industries such as telecommunication and technology will be highlighted. 

This study also contributes to the research by identify the current gaps in speech to text LLM, mainly in complex structured language like Japanese. This is achieved by offering a comparative analysis on which model is the best performance that can guide future improvement and innovation. This study aims to increase the effectiveness of speech to text adoption thus enhance the real-world application that rely on efficient transcription. 

\section{Conclusion}
In this chapter, the advancement in machine learning and artificial intelligence that made the computer can understand human better by improving the speech to text model accuracy and speed has been discussed. However, there is still challenges to transcribe a language that has complex structure like Japanese that include syllable-based formation and the use of multiple writing systems. Because of this, a study to find which implementation and which model is the most performance for handling Japanese language. The finding from this study is very important to answer the question of which model is the best for speech-to-text solution in Japanese language. By identifying the specific linguistic challenges and comparing these models, this study will provide a valuable information that will be able to guide future advancements in speech-to-text technology in Japanese language and ultimately will be able to support its broader application across the industries that rely heavily on precise and efficient transcription.

